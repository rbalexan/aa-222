\documentclass[conference]{IEEEtran}

\IEEEoverridecommandlockouts

\usepackage{cite}
\usepackage{amsmath,amssymb,amsfonts}
\usepackage{algorithmic}
\usepackage{graphicx}
\usepackage{textcomp}
\usepackage{xcolor}

\usepackage{booktabs} %@{}
\usepackage{pgfplots}
\pgfplotsset{compat=1.16}
\usepackage[per-mode=symbol,detect-all]{siunitx}
\usepackage{hyperref}
\usepackage{cleveref} %\Cref{} vs. \cref{}
\usepackage[protrusion=true,expansion=true]{microtype}
\usepackage{mathabx} % for \bigtimes


\def\BibTeX{{\rm B\kern-.05em{\sc i\kern-.025em b}\kern-.08em
    T\kern-.1667em\lower.7ex\hbox{E}\kern-.125emX}}

\begin{document}


\title{\LARGE \textbf{Project 1 -- Unconstrained Optimization} 
%\thanks{Ross Alexander is supported by a Stanford Graduate Fellowship (SGF) in Science and Engineering.}
}


\author{\IEEEauthorblockN{  Ross Alexander}
\IEEEauthorblockA{\textit{  Department of Aeronautics and Astronautics} \\
\textit{                    Stanford University} \\
                            Stanford, CA 94305 \\
                            rbalexan@stanford.edu}} % or ORCID


\maketitle

\begin{abstract}
    %
\end{abstract}

% \begin{IEEEkeywords}
% component, formatting, style, styling, insert
% \end{IEEEkeywords}

\section{Introduction}
\label{sec:introduction}

% Overview of topic sentences

% What is the problem?                     

% Why is it interesting and important?
% Why is it hard? Why do naive approaches fail?

% Why hasn't it been solved before? (Or, what's wrong with previous proposed solutions? How does mine differ?)

% What are the key components of my approach and results? Also include any specific limitations.

% Summary of the major contributions in bullet form, mentioning in which sections they can be found. This material doubles as an outline of the rest of the paper, saving space and eliminating redundancy.

Describe which algorithm(s) you chose and how they work. A couple of sentences could be sufficient. Convergence over one run is sufficient as well, yes. Describing the plots is only necessary if they're not self explanatory.

In addition to the programming aspect, you are also required to submit (also on gradescope) a PDF writeup, worth 50\% of the assignment. It should contain the following information:- A description of the method(s) you chose.

- A plot showing the path for Rosenbrock’s function with the objective contours and the path taken by your algorithm from three different starting points of your choice.
- Convergence plots for the three simple functions (Rosenbrock’s function, Himmelblau’s function, and Powell’s function).

- You can base your algorithm on those found in the book or online, but you must give credit.

- 

if prob == "simple1" || prob == "simple2" || prob == "simple3"
        descent\_method = Adam(2e-1, 0.7, 0.99, 1e-6, 0, 0, 0)
    else
        descent_method = Adam(3e-1, 0.7, 0.99, 1e-6, 0, 0, 0)

\end{document}